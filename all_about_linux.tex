\documentclass{book}
\raggedbottom
\usepackage[a4paper, total={6.5in, 8in}]{geometry}
\usepackage{enumitem}

\usepackage{fullpage}
\begin{document}
    \begin{center}
        \vspace*{\stretch{1}}
            {\fontsize{24.88}{5} \selectfont All About Linux}
        \vspace*{\stretch{1}}
    \end{center}

    \pagebreak

    \begin{flushleft}
        \noindent {\textbf{\fontsize{18}{0.72} \selectfont 0. Purpose}} \\[10pt]
    \end{flushleft}
    \noindent There is an incredible amount of information about the Linux kernel, but I have found
    it is either scattered or outdated. Many great books such as The Linux Kernel Interface,
    and Linux Kernel Programming exist, however, they discuss relatively outdated infromation.  \\

    \noindent The aim of this book is to consolidate information about the latest versions of the kernel.
    I intend this book to be a comprehensive, one-stop reference for all things Linux --- from process
    scheduling to memory management to disk IO. Where necessary, core computer architecture and organization,
    and TCP/IP networking will be delved into. \\

    \noindent The information in this book is sourced from:
        \begin{itemize}
            \setlength{\itemsep}{0.5pt plus 1pt}
            \item the Linux source code
            \item articles on the Internet
            \item classic books (such as ones previously alluded to in the previous section)
            \item man pages
            \item conference talks
        \end{itemize}

    \pagebreak

    \begin{flushleft}
        \noindent {\textbf{\fontsize{18}{18} \selectfont 1. Processes}} \\[10pt]
    \end{flushleft}

    \pagebreak

    \begin{flushleft}
        \noindent {\textbf{\fontsize{18}{18} \selectfont 2. Process Scheduling}} \\[10pt]
    \end{flushleft}

    \pagebreak

    \begin{flushleft}
        \noindent {\textbf{\fontsize{18}{18} \selectfont 3. Memory Management}} \\[10pt]
    \end{flushleft}

    \pagebreak

\end{document}